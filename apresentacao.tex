\documentclass[10pt]{beamer}
\usetheme[
%%% options passed to the outer theme
%    hidetitle,           % hide the (short) title in the sidebar
%    hideauthor,          % hide the (short) author in the sidebar
%    hideinstitute,       % hide the (short) institute in the bottom of the sidebar
%    shownavsym,          % show the navigation symbols
%    width=2cm,           % width of the sidebar (default is 2 cm)
%    hideothersubsections,% hide all subsections but the subsections in the current section
%    hideallsubsections,  % hide all subsections
%    left                % right of left position of sidebar (default is right)
  ]{Aalborg}

% If you want to change the colors of the various elements in the theme, edit and uncomment the following lines
% Change the bar and sidebar colors:
%\setbeamercolor{Aalborg}{fg=red!20,bg=red}
%\setbeamercolor{sidebar}{bg=red!20}
% Change the color of the structural elements:
%\setbeamercolor{structure}{fg=red}
% Change the frame title text color:
%\setbeamercolor{frametitle}{fg=blue}
% Change the normal text color background:
%\setbeamercolor{normal text}{bg=gray!10}
% ... and you can of course change a lot more - see the beamer user manual.

\usepackage[utf8]{inputenc}
\usepackage[brazil]{babel}
\usepackage[T1]{fontenc}
\usepackage{natbib}
% Or whatever. Note that the encoding and the font should match. If T1
% does not look nice, try deleting the line with the fontenc.
\usepackage{helvet}

% ---
% Extras
% ---
\usepackage{booktabs}

% colored hyperlinks
\newcommand{\chref}[2]{%
  \href{#1}{{\usebeamercolor[bg]{Aalborg}#2}}%
}
\setbeamerfont{caption}{size=\tiny}
\setbeamercolor{postit}{fg=white,bg=beamer@headercolor}
\setbeamercolor{alert}{fg=white,bg=red}

\title[]% optional, use only with long paper titles
{Título do trabalho}

\subtitle{Sub-título}  % could also be a conference name

\date{\today}

\author[] % optional, use only with lots of authors
{
  Nome\\
  \href{mailto:email@domain.com}{{\tt email@domain.com}}
}
% - Give the names in the same order as they appear in the paper.
% - Use the \inst{?} command only if the authors have different
%   affiliation. See the beamer manual for an example

\institute[
 {\includegraphics[width=1cm]{figuras/logo-unb}}\\ %insert a company, department or university logo
  Engenharia de Software\\
  Faculdade\ do Gama\\
  Universidade de Brasília
] % optional - is placed in the bottom of the sidebar on every slide
{% is placed on the bottom of the title page
  Engenharia de Software\\
  Faculdade do Gama\\
  Universidade de Brasília

  %there must be an empty line above this line - otherwise some unwanted space is added between the university and the country (I do not know why;( )
}

% specify the logo in the top right/left of the slide
% \pgfdeclareimage[height=1cm]{mainlogo}{figuras/logo} % placed in the upper left/right corner
% \logo{\pgfuseimage{mainlogo}}

% specify a logo on the titlepage (you can specify additional logos an include them in
% institute command below
\pgfdeclareimage[height=1.5cm]{titlepagelogo}{figuras/logo-unb} % placed on the title page
% \pgfdeclareimage[height=1.5cm]{titlepagelogo2}{figuras/logo-fga} % placed on the title page
\titlegraphic{% is placed on the bottom of the title page
  \pgfuseimage{titlepagelogo}
 % \hspace{5cm}\pgfuseimage{titlepagelogo}
}


% ---
% Início do documento
% ---
\begin{document}
% the titlepage
{\aauwavesbg
\begin{frame}[plain,noframenumbering] % the plain option removes the sidebar and header from the title page
  \titlepage
\end{frame}}
%%%%%%%%%%%%%%%%
\input{fixos/agenda}

% ------------------------------------------------------------
% Inserir slides aqui
% ------------------------------------------------------------
\section{Macro Tópico}

\begin{frame}{Macro Tópico}{}
\begin{block}{Slide 1}
  \begin{itemize}
    \item<1-> Item 1 do slide, aparece primeiro.
    \item<2-> Item 2 e item 3
    \item<2-> aparecem juntos.
    \item<3-> O \'<n->\' indica a ordem de exibição.
  \end{itemize}
\end{block}
\end{frame}
%%%%%%%%%%%%%%%%

\begin{frame}{Macro Tópico}
\begin{block}{Slide 2}
  \begin{itemize}
    \item<1-> Itens de outro slide.
  \end{itemize}
\end{block}
\end{frame}
%%%%%%%%%%%%%%%%


% ----------------- Referências --------------------------------

% --- O comando \allowframebreaks ---
% Se o conteúdo não se encaixa em um quadro, a opção allowframebreaks instrui
% beamer para quebrá-lo automaticamente entre dois ou mais quadros,
% mantendo o frametitle do primeiro quadro (dado como argumento) e acrescentando
% um número romano ou algo parecido na continuação.

\begin{frame}{Referências}
  \nocite{*}
  \bibliographystyle{plain}
  \bibliography{editaveis/bibliografia}
\end{frame}

% ----------------- FIM DO DOCUMENTO -----------------------------------------

{\aauwavesbg%
\begin{frame}[plain,noframenumbering]%
  \finalpage{Obrigado!}
\end{frame}}
%%%%%%%%%%%%%%%%

\end{document}
